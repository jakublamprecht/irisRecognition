\chapter{Podsumowanie i kierunki dalszego rozwoju}

Głównym celem pracy było wytworzenie aplikacji pozwalającej użytkownikowi na przeprowadzenie
procesu rozpoznawania tęczówki wraz z możliwością kontroli użytych metod oraz paramterów. W pracy
zaimplementowane zostały algorytmy zaproponowane przez Johna Daugmana \cite{DaugmanHowIrisRecognitionWorks}:
normalizacja metodą ``Rubber sheet model'', kodowanie z wykorzystaniem jednowymiarowych filtrów
Gabora oraz dopasowanie z wykorzystaniem odległości Hamminga.\newline

Kolejnym wymaganiem było stworzenie przejrzystego interfejsu użytkownika. W tym celu podzielono
system na dwie aplikacje - aplikację wykonującą przetwarzanie i obliczenia (napisaną w Pythonie z
wykorzystaniem biblioteki OpenCV) oraz aplikację kliencką pozwalającą użytkownikowi kontrolowanie
procesu oraz prezentację wyników (stworzoną w technologiach internetowych).\newline

Stworzenie zaawansowanego interfejsu użytkownika od podstaw oraz zapewnienie komunikacji z aplikacją
przetwarzającą było dużym wyzwaniem z uwagi na dużą iloś\'c wyprodukowanego kodu oraz potencjalną
błędogennoś\'c wynikającą z liczności widoków aplikacji. Dodatkowym czynnikiem utrudniającym
implementajcę systemu była integracja z istniejącymi algorytmami segmentacji.\newline

W końcowym etapie pracy przeprowadzony został eksperyment sprawdzający jakoś\'c działania
systemu z użyciem różnych wartości parametrów dla poszczególnych metod. Za wartości domyślne
przyjęte zostały wartości zaproponowane przez Maseka w jego pracy \cite{masek}. Wyniki
eksperymentu wykazały bardzo dobrą jakoś\'c działania systemu, w tym prawdopodobieństwo
błędnego zaakceptowania identyfikowanej tęczówki na poziomie 0.0001.\newline

\noindent
Poniżej przedstawione zostały potencjalne kierunki dalszego rozwoju niniejszej pracy:

\begin{itemize}
  \item Implementacja kolejnych metod normalizacji, kodowania oraz dopasowania,
  \item Poprawa jakości oraz szybkości działania algorytmów segmentacji,
  \item Zapewnienie możliwości kodowania tęczówki za pomocą wielu filtrów Gabora
  \item Poprawa komunikatów o błędach w aplikacji
  \item Stworzenie widoku historii operacji dla trybu krokowego
  \item Adaptacja aplikacji klienckiej i przetwarzającej w celu przeniesienia
  aplikacji przetwarzającej na system mikroprocesorowy.
  \item Dodanie integracji z bazą danych
\end{itemize}
