\chapter{Wprowadzenie}

% [DW] Cytowania normalnie, bez dodatkowych nawiasów; poprawiłem niektóre, zrób pozostałe.

Ogolny opis pracy - wstep do tematyki + motywacja wyboru tematu.

\section{Cel i~zakres pracy}

Celem tej pracy było zaprojektowanie i stworzenie aplikacji pozwalającej na przeprowadzenie procesu
rozpoznawania tęczówki oka oraz przegląd literatury i implementacja algorytmów normalizacji,
enkodowania oraz dopasowania tęczówki. Jednym z wymagań dla tworzonego programu, było aby działał
on w dwóch trybach:

\begin{itemize}
    \item Krokowym - tryb umożliwiający użytkownikowi stopniowe przejście przez proces rozpoznawania
    z możliwością wyboru metod oraz parametrów dla poszczególnych kroków, a także podglądu wyników dla
    każdego z nich.

    \item Wsadowym - tryb umożliwiający automatyczne przeprowadzenie procesu rozpoznawania opisanego przez
    plik konfiguracyjny wygenerowany w trybie krokowym dla większego zbioru obrazów.
\end{itemize}

\noindent
Ważnym aspektem dla tworzonej aplikacji była również użyteczność i przyjazność interfejsu użytkownika,
który miał zachęcać do korzystania z aplikacji.

\section{Struktura pracy}

Opis poszczególnych rozdziałów - zrób jak już wymyślisz rozdziały.

