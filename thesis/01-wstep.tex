\chapter{Wprowadzenie}

% [DW] Cytowania normalnie, bez dodatkowych nawiasów; poprawiłem niektóre, zrób pozostałe.

Wraz ze zwiększeniem stopnia informatyzacji na świecie zwiększa się również iloś\'c przechowywanych przez
systemy informatyczne danych, w tym również tych poufnych, do których dostęp powinny mie\'c tylko
określone osoby. W związku z tym zapotrzebowanie na systemy weryfikacji i identyfikacji rośnie. Metody
uwierzytelniania można podzieli\'c ze względu na element użyty do przeprowadzenia procesu:

\begin{itemize}
  \item metody wykorzystujące informacje, do których dostęp mają wyłącznie osoby uprawnione (np. hasło, pin),
  \item metody wykorzystujące przedmiot w którego posiadaniu są jedynie osoby uprawnione,
  \item metody wykorzystujące cechy fizyczne osób uprawnionych.
\end{itemize}

Z wymienionych metod najbezpieczniejszymi są te ostatnie, inaczej zwane metodami biometrycznymi.
Twórcy systemów wykorzystujących te metody poszukują cech fizycznych człowieka, które pozwalają na
nieinwazyjne, odporne na oszustwa uwierzytelnienie. Jedną z najbardziej obiecujących cech
w świecie biometrii jest tęczówka, której rozpoznawanie jest przedmiotem niniejszej pracy. Mimo, że
tęczówka nie jest cechą idealną ze względu na przykładowo koszty wdrożenia takiego systemu,
to systemy wykorzystujące tę cechę biometryczną charakteryzują się niskim współczynnikiem nieprawidłowej
identyfikacji.

\section{Cel i~zakres pracy}

Celem tej pracy było zaprojektowanie i stworzenie aplikacji pozwalającej na przeprowadzenie procesu
rozpoznawania tęczówki oka oraz przegląd literatury i implementacja algorytmów normalizacji,
kodowania oraz dopasowania tęczówki. Jednym z wymagań dla tworzonego programu było, aby działał
on w dwóch trybach:

\begin{itemize}
    \item Krokowym - tryb umożliwiający użytkownikowi stopniowe przejście przez proces rozpoznawania
    z możliwością wyboru metod oraz parametrów dla poszczególnych kroków, a także podglądu wyników dla
    każdego z nich.

    \item Wsadowym - tryb umożliwiający automatyczne przeprowadzenie procesu rozpoznawania opisanego przez
    plik konfiguracyjny wygenerowany w trybie krokowym dla większego zbioru obrazów.
\end{itemize}

\noindent
Ważnym aspektem dla tworzonej aplikacji była również użyteczność i przyjazność interfejsu użytkownika,
który miał zachęcać do korzystania z aplikacji.

\section{Struktura pracy}

Struktura pracy jest następująca. Rozdział drugi poświęcony został wstępowi do tematu
rozpoznawania tęczówki. Zawiera on ogólne informacje o historii zagadnienia, anatomii oka oraz właściowościach
i zastosowaniach biometrycznych tęczówki. Traktuje on również ogólnie o procesie
przetwarzania obrazu tęczówki. W kolejnym rozdziale opisane zostały technologie użyte w celu stworzenia
aplikacji, a także opis jej działania i funkcjonalności. Rodział czwarty jest w całości poświęcony
algorytmom normalizacji, kodowania oraz dopasowania i zasadzie ich działania. Piąty rozdział
przedstawia informacje na temat eksperymentu przeprowadzonego z wykorzystaniem wytworzonej
aplikacji oraz jego wyniki. Rozdział szósty stanowi podsumowanie pracy i wskazuje potencjalne
kierunki dalszego jej rozwoju.

