\section*{Streszczenie}

Celem niniejszej pracy dyplomowej była implementacja oraz analiza algorytmów normalizacji,
kodowania oraz dopasowania w procesie rozpoznawania tęczówki. W ramach pracy powstała aplikacja
pozwalająca na przeprowadzenie pełnego procesu rozpoznawania w trybie \textbf{krokowym} pozwalającym
na kontrolowanie użytych metod oraz paramterów, a także w trybie \textbf{wsadowym} pozwalającym na przeprowadzenie
procesu dla większego zbioru obrazów. W pracy zaimplementowane oraz opisane zostały algorytmy zaproponowane przez
Johna Daugmana. W końcowej części pracy przeprowadzony został eksperyment sprawdzający jakoś\'c
działania zaimplementowanego systemu poprzez obliczenie współczynników \textbf{FAR} \textit{(ang. False Acceptance Rate)}
oraz \textbf{FRR} \textit{(ang. False Rejection Rate)}. Ważnym aspektem pracy było stworzenie
przejrzystego interfejsu użytkownika, w związku z czym do jego implementacji użyto technologii
internetowych w połączeniu z OpenCV.

\section*{Abstract}

The purpose of this dissertation was to implement and analyse normalisation, feature extraction
and matching algorithms in the process of iris recognition. An application capable of performing
the whole process of iris recognition has been developed as a part of this paper. Said application
provides users with two modes - \textbf{step mode} which allows user to preview the results and control used
methods and parameters in each step of the process and \textbf{batch mode} which provides the user
with possibility of running the process on a bigger set of images. An implementation and analysis
of John Daugman algorithms has been the focus of this paper. Lastly an accuracy check of the implemented
system has been performed by calculating \textbf{FAR} \textit{(False Acceptance Rate)} and \textbf{FRR} \textit{(False Rejection Rate)}.
Providing a user-friendly interface was an important part of created application, which is why
web technologies have been integrated with Python and OpenCV powered backend application.
